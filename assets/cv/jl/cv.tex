%-------------------------
% Resume in Latex
% Author : Joohyun Lee
% Website: 
%------------------------

\documentclass[letterpaper,10pt]{article}
\usepackage{charter}
\usepackage{geometry}
\usepackage{tabularx}

\usepackage{textcomp}
\usepackage{latexsym}
\usepackage[empty]{fullpage}
\usepackage{titlesec}
\usepackage{marvosym}
\usepackage[usenames,dvipsnames]{color}
\usepackage{verbatim}
\usepackage{enumitem}
\usepackage[pdftex]{hyperref}
\usepackage{fancyhdr}

\pagestyle{fancy}
\fancyhf{} % clear all header and footer fields
\fancyfoot{}
\renewcommand{\headrulewidth}{0pt}
\renewcommand{\footrulewidth}{0pt}

% Adjust margins
\oddsidemargin=-0.2in
\evensidemargin=1in
\footskip=5pt
\addtolength{\textwidth}{0.4in}
\addtolength{\topmargin}{-0.3in}
\addtolength{\textheight}{1in}
\addtolength{\voffset}{-0.0in}

\urlstyle{same}

\raggedbottom
\raggedright
\setlength{\tabcolsep}{0in}

\definecolor{myblue}{RGB}{31, 119, 180}

% Sections formatting \scshape
\titleformat{\section}{
  \vspace{-4pt}\raggedright\large\color{myblue}
}{}{0em}{}[\color{black}\titlerule \vspace{-5pt}]

%-------------------------
% Custom commands
\newcommand{\resumeItem}[2]{
  \item\small{
    \textbf{#1}{: #2 \vspace{-2pt}}
  }
}

\newcommand{\resumeSubheading}[4]{
  \vspace{-1pt}\item
    \begin{tabular*}{0.97\textwidth}{l@{\extracolsep{\fill}}r}
      \textbf{#1} & #2 \\
      {\small #3} & {\small #4} \\
    \end{tabular*}\vspace{-5pt}
}

\newcommand{\resumeSubItem}[2]{\resumeItem{#1}{#2}\vspace{-4pt}}

\renewcommand{\labelitemii}{$\circ$}

\newcommand{\resumeSubHeadingListStart}{\begin{itemize}[leftmargin=*,label={}]}
\newcommand{\resumeSubHeadingListEnd}{\end{itemize}}
\newcommand{\resumeItemListStart}{\begin{itemize}}
\newcommand{\resumeItemListEnd}{\end{itemize}\vspace{-5pt}}

% Custom adjustments
\usepackage{setspace}
\onehalfspacing
\newcommand{\tabitem}{{\textbullet}~}

%-------------------------------------------
%%%%%%  CV STARTS HERE  %%%%%%%%%%%%%%%%%%%%%%%%%%%%


\begin{document}{\rightline{Curriculum Vitae -- Joohyun Lee}}

\rule{1\textwidth}{1.5pt}
\vspace{1pt}

%----------HEADING-----------------
\begin{tabular*}{\textwidth}{l@{\extracolsep{\fill}}r}
  \textbf{\LARGE Joohyun Lee} & \\
  \\[-11pt]
  The University of Texas at Austin, &
  \href{https://joohyun-lee.github.io/}{https://joohyun-lee.github.io/}\\
  PMA 16.212, Astronomy Department, & 
  Email : \href{mailto:jhl1862@gmail.com}{jhl1862@gmail.com}, \href{mailto:jhl1862@austin.utexas.edu}{joohyun.lee@austin.utexas.edu}\\
  2515 Speedway, Austin, Texas 78712-1205 &
  \href{https://orcid.org/0000-0001-8593-8222}{https://orcid.org/0000-0001-8593-8222}
  %Phone : (+82) 10-2291-1862\
\end{tabular*}


%-----------RESEARCH INTEREST-----------------
\section{\textbf{RESEARCH INTEREST}}
\begin{center}
\begin{tabular*}{0.97\textwidth}{l}
  %\textbf{Theoritical \& Computational Astrophysics}\\
  %high-resolution simulation of galaxy formation \& evolution, merger, AGN-host galaxy co-evolution;\\ galaxy formation \& evolution in cosmological simulation\\
  %numerical method, high performance computing for cosmological simulation\\
  \textbf{Theoretical \& Computational Astrophysics}\\
  numerical cosmological simulation of the epoch of reionization;\\
  role of dark matter models in the growth of structures;\\
  general galaxy formation \& evolution; usage of machine learning in simulation data analysis\\
\end{tabular*}
\end{center}


\vspace{-15pt}


%-----------EDUCATION-----------------
\section{\textbf{EDUCATION}}
\begin{center}
\begin{tabular*}{0.97\textwidth}{l@{\extracolsep{\fill}}r}
  \textbf{Ph.D. in Astronomy,} The University of Texas at Austin & 09/2021 - present\\
  \hspace{12pt}\textit{Supervisor: Paul Shapiro}
\end{tabular*}
\end{center}

\vspace{-20pt}

\begin{center}
\begin{tabular*}{0.97\textwidth}{l@{\extracolsep{\fill}}r}
  \textbf{B.Sc. in Physics \& B.Eng. in Electrical and Computer,} Seoul National University & 03/2014 - 08/2021\\
  %\hspace{11pt}\textit{\small Overall GPA : 3.78/4.3 (8.8/10)} & (intended)\\
\end{tabular*}
\end{center}


\vspace{-15pt}


%\begin{center}
%\begin{tabular*}{0.97\textwidth}{l@{\extracolsep{\fill}}r}
  %\textbf{Graduated First Class Honor,} {\small Daejeon Science High School} & 03/2012 - 02/2014\\
%\end{tabular*}
%\end{center}


%-----------RESEARCH EXPERIENCE-----------------
\section{\textbf{RESEARCH EXPERIENCE}}

\begin{center}
\begin{tabular*}{0.97\textwidth}{l@{\extracolsep{\fill}}r}
    \textbf{Graduate Student Fellow,} The University of Texas at Austin & 
    09/2021 - \\
    \textit{\small (Supervisor: Prof. Paul Shapiro)}
\end{tabular*}
\end{center}

\vspace{-20pt}

\begin{center}
\begin{tabular*}{0.97\textwidth}{l}
  \tabitem \underline{Multiple Beads-on-a-string Dark-Matter-Deficient Galaxies Produced in a Mini-Bullet (Cluster) Galaxy Collision}\\
  \hspace{5pt}{\small - Ran a suite of $N$-body/hydro simulations of Satellite-satellite galaxy collisions with the presence of a massive host}\\
  \hspace{5pt}{\small - Performed orbit integration of the produced dark-matter-deficient galaxies to compare with observations}\\
\end{tabular*}
\end{center}


\vspace{-15pt}


\begin{center}
\begin{tabular*}{0.97\textwidth}{l@{\extracolsep{\fill}}r}
  %\textbf{Research Intern, Photonic Systems Laboratory,} {\small Seoul National University} & 2016\\
  %\textit{\small with Prof. Namkyoo Park}\\
  %\\[-12pt]
  \textbf{Research Associate, Computational Cosmology Group,} Seoul National University & 
  09/2019 - 08/2021\\
  \textit{\small (Supervisor: Prof. Ji-hoon Kim)}\\
\end{tabular*}
\end{center}

\vspace{-20pt}

\begin{center}
\begin{tabular*}{0.97\textwidth}{l}
  \tabitem \underline{Estimating Galactic Baryonic Properties from Their Dark Matter Using Machine Learning}\\
  \hspace{5pt}{\small - Applied trained machine to the cosmological simulation halo catalog (IllustrisTNG simulation)}\\
  \hspace{5pt}{\small - Computed and compared two-point correlation function in IllustrisTNG halo catalog and machine-predicted halo catalog}\\
  \tabitem \underline{Dark-Matter-Deficient Galaxies Produced Via High-velocity Galaxy Collision in Cosmological Simulation}\\
  \hspace{5pt}{\small - Studied IllustrisTNG catalog to find high-speed collision events of dwarf galaxies to compare with idealized simulation}\\
  \tabitem \underline{pc-scale Simulation of Simultaneous Formation of Dark-Matter-Deficient Galaxies and Star Clusters}\\
  \hspace{5pt}{\small - Ran a suite of 1.25 pc-resolution galaxy collision simulations with different merger configuration and feedback schemes}\\
  \hspace{5pt}{\small - Resolved and tracked the formation process of dark-matter-deficient galaxies and massive star clusters}\\
\end{tabular*}
\end{center}


\vspace{-15pt}


\begin{center}
\begin{tabular*}{0.97\textwidth}{l@{\extracolsep{\fill}}r}
    \textbf{Research Associate, AGN Research Group,} Seoul National University & 
    09/2020 - 02/2021\\
    \textit{\small (Supervisor: Prof. Jong-Hak Woo)}
\end{tabular*}
\end{center}

\vspace{-20pt}

\begin{center}
\begin{tabular*}{0.97\textwidth}{l}
  \tabitem \underline{Calibrated and Applied Novel Method of Measuring SFR in AGNs}\\
  \hspace{5pt}{\small - Tested Oxygen emission line flux as SFR indicator by statistically analyzing SDSS spectroscopy data and IR surveys}\\
  \hspace{5pt}{\small - Investigated correlation between gas outflow strength from AGNs and star formation of host galaxies}\\
\end{tabular*}
\end{center}


\vspace{-15pt}


%-----------FELLOWSHIP-----------------
\section{\textbf{AWARDED FELLOWSHIPS \& SCHOLARSHIPS}}
\begin{center}
\begin{tabular*}{0.97\textwidth}{l@{\extracolsep{\fill}}r}
  \textbf{FINESST Fellowship} (\boldmath$\$150{\rm k}$), NASA &
  09/2022 - 08/2025\\
  \textbf{Dean’s Excellence Fellowship}, University of Texas at Austin &
  09/2021 - 08/2022\\
  \textbf{Presidential Science Scholarship} (\boldmath$\sim \$40{\rm k}$), Korea Student Aid Foundation & 
  03/2014 - 08/2020\\
\end{tabular*}
\end{center}



\newpage

%-----------PUBLICATIONS-----------------
\section{\textbf{PUBLICATIONS}}
\begin{center}
\begin{tabularx}{0.97\textwidth}{X}
\tabitem {\small \textbf{Lee, J.}, Shin, E. -j., Kim, J. -h., Shapiro, P. R., \& Chung, E.,
  ``Multiple Beads-on-a-string: Dark Matter-Deficient Galaxy Formation in a Mini-Bullet Satellite-satellite Galaxy Collision'',
  {\textit{ApJ Accepted}}, 
  \href{https://arxiv.org/abs/2312.11350}{\textit{astro-ph:2312.11350}}}

\tabitem {\small \textbf{Lee, J.}, Shin, E. -j., \& Kim, J. -h., 
  ``Dark Matter Deficient Galaxies And Their Member Star Clusters Form Simultaneously During High-velocity Galaxy Collisions In 1.25 pc Resolution Simulations'',
  \href{https://iopscience.iop.org/article/10.3847/2041-8213/ac16e0}{\textit{ApJL 917 (2021) L15}}, 
  \href{https://arxiv.org/abs/2108.01102}{\textit{astro-ph:2108.01102}}}
  
\tabitem {\small Shin, E. -j., Jung, M., Kwon, G., Kim, J. -h, \textbf{Lee, J.}, Jo, Y., \& Oh, B. K.,
  ``Dark Matter Deficient Galaxies Produced Via High-velocity Galaxy Collisions In High-resolution Numerical Simulations'',
  \href{https://iopscience.iop.org/article/10.3847/1538-4357/aba434/meta}{\textit{ApJ 899 (2020) 25}},
  \href{https://arxiv.org/abs/2007.09889}{\textit{astro-ph:2007.09889}}}
\end{tabularx}
\end{center}



%-----------TALKS-----------------
%Add talk titles
\section{\textbf{CONTRIBUTED TALKS \& PRESENTATIONS}}
\begin{center}
\begin{tabularx}{0.97\textwidth}{l@{\extracolsep{\fill}}r}
  \tabitem UT Austin Extragalactic/Cosmology Seminar & 11/2023\\
  \tabitem FirstLight Conference (poster) & 06/2023\\
  \tabitem UT Austin Extragalactic/Cosmology Seminar & 05/2023\\
  \tabitem IAUGA 2022 (poster) & 08/2022\\
  \tabitem APS April Meeting 2022 & 04/2022\\
  \tabitem Galaxy Evolution Workshop 2021, ASIAA & 02/2022\\
  \tabitem Numerical Galaxy Formation Mini-Workshop, SNU & 01/2022\\
  \tabitem SAZERAC-SIPS Early Galaxy Formation Near and Far --- Preparing for a Long Journey with JWST & 12/2021\\
  \tabitem The 1st KIAA Forum on Gas in Galaxies for Early Career Scientists (KooGiG-Junior workshop) & 10/2021\\
  \tabitem UT Austin Extragalactic/Cosmology Seminar & 09/2021\\
  \tabitem AGORA WORKSHOP 2021 & 08/2021\\
\end{tabularx}
\end{center}


%-----------COMPUTING SKILLS and EXPERIENCES-----------------
\section{\textbf{COMPUTING SKILLS \& EXPERIENCES}}
\begin{center}
\begin{tabular*}{0.97\textwidth}{l}
  \textbf{Languages}: Python, C, C++, LaTex (skilled); Fortran, MATLAB, Mathematica, HTML, Markdown (familiar);\\
  \hspace{0.73in} IDL, RISC-V assembly language (basic)\\
  \textbf{Astrophysical Simulation Codes}: Enzo, Ramses, Gadget, MUSIC, DICE, yt\\
  \textbf{Machine Learning}: JAX, PyTorch (familiar); TensorFlow (basic)\\
  \textbf{High performance computing experience}:\\
  \tabitem Local cluster of Computational Cosmology Group, Seoul National University (CentOS)\\
  \tabitem Nurion, Korea Institute of Science and Technology Information (CentOS)\\
  \tabitem Frontera (CentOS), Stampede3, Stampede2 (Red Hat), Texas Advanced Computing Center \\
  \tabitem Andes, Oak Ridge National Laboratory (Linux)\\
\end{tabular*}
\end{center}


%-----------GRADUATE LEVEL COURSES-----------------
%\section{\textbf{GRADUATE LEVEL COURSES}}
%\begin{center}
%\begin{tabular*}{0.97\textwidth}{l}
  %\textbf{Astrophysics}: Interstellar Medium, Extragalactic Astronomy \& Cosmology, 
  %\\\hspace{2.35cm}
  %Modern Cosmology, Interstellar Gas Dynamics (now taking)\\
  %\textbf{Physics}: Classical Mechanics
%\end{tabular*}
%\end{center}


%-----------CONFERENCE and EXTERNAL SCHOOL-----------------
%\section{\textbf{CONFERENCES and EXTERNAL SCHOOL}}
%\begin{center}
%\begin{tabular*}{0.97\textwidth}{l@{\extracolsep{\fill}}r}
  %SLAC Summer Institute 2020 & 08/2020\\
  %\\[-12pt]
  %2020 KIAS Astrophysics Summer School, Korea Institute for Advanced Study (KIAS) & 07/2020\\
  %\\[-12pt]
  %Numerical Galaxy Formation Mini-Workshop, Seoul National University & 02/2020\\
  %\\[-12pt]
  %2019 KIAS-SNU Winter Camp on Theoretical Physics, Korea Institute for Advanced Study (KIAS) & 01/2020\\
  %\\[-12pt]
  %Nishina School 2019, RIKEN Nishina Center for Accelerator-Based Science & 08/2019
%\end{tabular*}
%\end{center}


%-----------OUTREACH and TEACHING EXPERIENCES-----------------
\section{\textbf{MENTORING, OUTREACH \& TEACHING EXPERIENCES}}
\begin{center}
\begin{tabular*}{0.97\textwidth}{l@{\extracolsep{\fill}}r}
  GUMMY Mentor in UT Astro. Dept. & 08/2023 - \\
  Informal Mentor of NSF REU Scholars & 06 - 08/2022 \& 06 - 08/2023\\
  Korea Student Aid Foundation Science Teaching Service Organization & 01 - 02/2015\\
  Habitat for Humanity in Cebu, Phillippines & 02/2016\\
  Military Service at Korean Air Force 5th Air Mobility Wing & 05/2017 - 04/2019\\
\end{tabular*}
\end{center}

%-----------OTHER SKILLS and INTERESTS-----------------
\section{\textbf{OTHER SKILLS}}
\begin{center}
\begin{tabularx}{0.97\textwidth}{X}
  Languages: Korean (native), English, Japanese (fluent)\\
  %Sports: soccer (football), basketball, table tennis\\
  %Interests: photography
\end{tabularx}
\end{center}

%-----------PUBLICATIONS-----------------
%\section{\textbf{PUBLICATIONS}}
%\begin{center}
%\begin{tabularx}{0.97\textwidth}{X}

%\end{tabularx}
%\end{center}


%-------------------------------------------
\end{document}
